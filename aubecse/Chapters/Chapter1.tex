% Chapter 1

\chapter{INTRODUCTION} % Write in your own chapter title All Chapter headings in CAPS
\section{General Overview} % All Section headings in Title Case
Robotics is an emerging technology in the field of science. These days many types of wireless robots are being developed and are put to varied applications and uses. The broad objective of taking up this project is to create a intuitive and innovative method of controlling a remote object using simple gestures. This approach is intuitive because the usage of hand gestures comes naturally to any user without prior training unlike generic remote controls. The innovativeness lies in the fact that while implementing the project, we used technology that have never been used together to achieve this result. A gesture recognition device interfaced with a processing unit is placed at the user's location. A simple moving robot with four wheels is present at a remote location anywhere on the globe with the constraint being that internet connection is available. As the user makes simple gestures the device recognises them, processes them and transfers it to the robot over the internet. The robot then moves in an appropriate direction based on the gesture performed. For example if the user performs a simple swipe right gesture, the robot moves in the right direction. Thus, we have defined a set of gestures and the consequent robot movements based on the gesture.
\section{About the project} %All Sub Section headings in Title Case
The project is divided into three main modules. The first module is the gesture recognition device interfaced with the processing unit. Gesture recognition is achieved using the Microsoft Kinect for Windows which is a motion sensing input device. While even a generic computer web cam is capable of gesture recognition, it is not the cutting edge technology that we aim to use. It lacks the basic accuracy, efficiency and performance. On the other hand, competitors like LeapMotion controller have a smaller observation area and Myo by Thalmic Labs is more suitable for muscular movement. Kinect's greatest merit lies in its ability to handle both high quality image and signal processing simultaneously. Not only does it have a greater field of observation, but is also developer friendly and popular in the market. These features led us to believe that the Kinect is the best possible gesture recognition device available in the scope of our project.\linebreak \linebreak
 The second module is the robot which is present at the remote location. The robot is controlled by the Raspberry-Pi (RPi), a microcomputer which is interfaced with the DC motors using an L298N full bridge motor driver. For constructing the robot, we had other choices like the Arduino. However, compared to the Arduino which is a commonly used micro-controller, the RPI offers a whole platform to work on as it is a micro-computer. Since the RPI is also very cost-effective, the RPI is adopted as the robot controller. \linebreak \linebreak
 The third module is basically interfacing the gesture recognition device with the robot over the internet. Since, we want the robot to be controlled remotely over a long distance, the internet is the best choice to achieve this long range communication. From any location on the globe, we can control the robot provided there is internet connection available at the remote location. Other interfacing mechanisms like bluetooth provide a much shorter range of communication. One of the drawbacks of using the internet is that it is difficult to model the transmission efficiency at any time period, because the speed of the internet is never constant.


