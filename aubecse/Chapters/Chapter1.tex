% Chapter 1

\chapter{INTRODUCTION} % Write in your own chapter title All Chapter headings in CAPS
\section{General Overview} % All Section headings in Title Case
The Quadcopter UAV is one of the fastest growing technologies of the 21st century. It has a myriad of uses and is a field of intensive research and continuous development. The broad objective of this project is to build a drone which would be able to take pictures of topography or objects, construct a panorama (360 degree view of the space) and recreate a 3D model of the environment. It is a highly useful technology as it can be used in aerial photogrammetry, reconstructing topographical landscapes for archaeological excavations, reconstruction of buildings and taking panorama shots for movie-making etc.
The quadcopter is built from scratch and runs on the Raspberry Pi microcomputer with a Python interface. The image processing is done on-board the craft using OpenCV 3.1.0. The flight dynamics of the quadcopter are automatically stabilized using a PID controller coded in Python. The drone is controlled manually using an Android App with a virtual joystick interface for easy control.
The drone's flight and functionality is controlled via the App and the processing of images to produce the panorama and 3D model is done onboard by the Raspberry Pi. The visualization of the 3D reconstuction is viewed on the laptop.
\section{About the project} %All Sub Section headings in Title Case
The project is divided into three main modules. The first module is the quadcopter itself. \linebreak
The quadcopter runs on the Raspberry Pi 2 microcomputer which hosts a Debian-based OS (Raspbian Jessie) which allows the use of OpenCV and Python. 
It has four motors whose speeds are controlled using Pulse Width Modulation(PWM) technique. The flight dynamics including stability and orientation are detected using the IMU(Inertial Measurement Unit) Sensor which consists of an accelerometer, gyroscope and magnetometer. The accelerometer measures the acceleration and tilt while the gyroscope measures the angular velocity. The sensor data are fed to the RaspberryPi(RPi) controller which then calculates the current orientation and automatically computes the required orientation of the quadcopter using a PID (Proportional-Integral-Derivative) controller. The quadcopter is equipped with a camera module which takes pictures and feeds them to the RPi which then performs the image processing.
\linebreak
The second module performs image processing. It is divided into two parts – Panorama Stitching and 3D Model Reconstruction. The module has been implemented using OpenCV 3.1.0 with Python support. The quadcopter takes pictures using the camera module which are fed to th RPi. OpenCV is used to stitch the images together to create a panorama or a 360 degree view of the environment. Panorama stitching allows one to take individual images of a wide physical space which cannot be covered in a single shot, finds overlapping features, matches them and stitches them into one large, combined shot. The second part of the module deals with 3D reconstruction. The reconstruction algorithm also called Structure from Motion(SfM) algorithm is used to recreate 3D point clouds of buildings or topographies whose pictures are taken by the quadcopter. The point clouds are then visualized on a laptop using a browser. The advantage of using a quadcopter is that it can fly to inaccessible locations and take shots which would otherwise be very difficult to take. It can therefore be used for aerial survey and movie-making.
\linebreak
The third module is the Android App which is used to control the quadcopter. Usually quadcopters are controlled using Radio Controllers. But this project makes use of an Android app which will have other functions apart from simply controlling the flight of the drone i.e. by sending JSON-encoded signals using WiFi to the Python server running on the RPi. The app has a virtual joystick interface for simple user-interaction and control. 