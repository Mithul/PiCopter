% Chapter 2
\chapter{RELATED WORK} % Chapter Title in ALL CAPSacs
A significant amount of research and work has gone into the quadcopter. Algorithms for controlling the stability of the quadcopter such as PID(Proportional-Integral-Derivative) and Fuzzy Logic exist. With improvements to open source image processing libraries like OpenCV and the development of RaspberryPi microcomputer, the capability and use of drones for various activities have increased.
\newline \newline
 Alex G. Kendall et al had developed an on-board object-tracking control of a quadcopter with monocular vision \cite{kendall2014board} which made use of a RPi for processing along with a camera module and sensors as input to the OpenCV library whic performed object tracking based on the color of the objects. While our project also made use of the RPi and OpenCV for image processing, it focused largely on the flight stability and image processing modules and did not perform object tracking.
\newline \newline
Quadcopters are predominantly controlled using Radio Control joysticks. Since Android phones are ubiquitous now, a virtual joystick interface was developed to control the flight and the image-processing functions of the quadcopter as an Android app.
\newline 
\newline 
Since our quadcopter's functions were panorama stitching and 3D reconstruction, an interface was required to render the results.  A. Zul Azfar and Hazry Desa \cite{azfar2011simple} had developed a Graphical User Interface for monitoring the flight dynamics and orientation of a remotely-operated quadcopter using LABView software.The motion of the quadcopter could be captured on the GUI using LabVIEW software. The quadcopter’s processing was done on-board using Arduino microcontroller and the balancing and stability were managed using an IMU sensor. This was a path-breaking paper as it demonstrated how an unmanned aerial vehicle could be controlled from a remote station. The project did not concern itself with the stability of the drone or perform any specialized functions.
\newline \newline
 Priyanga M. and Raja Ramanan \cite{proc-disc-2009} had developed a UAV for Video Surveillance to prevent unauthorized soil-mining operations. It used a RPi for on-board processing of live feeds from a webcam. The videos could be viewed on a webpage in real-time using WiFi. The drone used a GPS to calculate position and follow the target. The project demonstrated how a drone could be used for practical and real-time applications such as surveillance and tracking with high accuracy. 
\newline \newline
 In our project, signals from the app were sent as JSON-encoded data using WiFi to the Python server running on the quadcopter. The quadcopter itself was controlled remotely via WiFi using the Android App.
\newline \newline
Image processing on-board the quadcopter had been implemented by Shilpashree et al \cite{ijarcce-disc-2015} in their paper, “Implementation of Image Processing on Raspberry Pi”. They had described the purpose of  image-enhancing algorithms to improve the quality of the images taken by the drone due to compromised image quality from unstable flight. Image-enhancing algorithms such as Rudin-Osher-Fatemi de-noising model (ROF) had been implemented. 
\newline \newline
Shawn McCann's thesis on “3D Reconstruction from Multiple Images” \cite{stan-disc-2015} had tried to identify various techniques for dense and sparse reconstructions using Structure from Motion(SfM) algorithms. He had demonstrated and implemented the complete flow of 3D Reconstruction using the open source Bundler software. Our project made use of VLFeat for SfM support and OpenCV which offers libraries for feature detection and matching. In both projects, bundle adjustment was used to minimize reprojection errors and generate sparse point clouds. While \cite{stan-disc-2015} rendered the point clouds using third-party tools(MeshLab), our project made use of an interactive JS library called, threeJS which renderered the point cloud on a browser.
\newline \newline
Argentim et al wrote a paper on PID, LQR and LQR-PID on a quadcopter platform \cite{argentim2013pid} compared the stability of the use of a PID, LQR or a PID tuned with a LQR loop. Our project used the PID controller loop as the LQR loop. Although it was robust and produced a very low steady state error, it had a big transition delay due to six feedback gains, which caused a lag in the time taken between sensing the angle and computing the output due to the limited computing power of the RPi.
\newline \newline
Drone technology has advanced enormously along the lines of photography, surveillance, mapping of terrains and delivery. The mobility combined with the computational power of the RPi from these projects motivated us to build one of our own.
\newline \newline
The next chapter describes the requirements analysis of the project.
