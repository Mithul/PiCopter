% Chapter 2
\chapter{RELATED WORK} % Chapter Title in ALL CAPSacs
Remote control of robotic systems has been applied in manufacturing, underwater manipulation, disaster recovery operations, tele-operations, space exploration, etc. With recent advances on the Internet technology, remote control of mobile robots presents new opportunities in resources sharing, long-distance learning, and remote experimentation. Kun Qian, Jie Niu and Hong Yang [1] developed a Gesture Based Remote Human-Robot Interaction System Using Kinect. They used complex algorithms like Camshaft Based hand tracking and Hand Motion Trajectory recognition to obtain the gestures from the kinect sensor. Then they constructed a bi-handed Cyton robot arm which would be controlled using the gestures recognized. While our project conceptually inclines to the objective of this paper, the fundamental difference is in the manner the robot is built. The gestures in [1] are literally being mimicked by the robotic arm and hence the gestures are limited to the robot's design. We aim to define custom gestures depending on the motion of the robot (4 wheeler). However, the preliminary architecture design is strikingly similar. While [1] uses a computer to control the robot, we aim to cut down the costs and use a micro-computer (RPI) instead.
Silas F. dos Reis Alves, Alvaro J. Uribe-Quevedo, Ivan Nunes da Silva and Humberto Ferasoli Filho [2] developed Pomodoro, a Mobile Robot Platform for Hand Motion Exercising. Pomodoro is basically a device for encouraging hand motion exercise through flexion/extension and ulnar/radial deviation movements. It however makes use of Leap Motion technology to recognise the gestures. Since the cost of acquiring and maintaining a mobile robot is expensive, the system was designed with a complimentary 3D virtual environment, which did not requires the physical robot, but allowed connectivity with the real robot if available. The leap motion sensor tracking is prone to errors when the hand and/or fingers are occluded, or when there is no line- of-sight between the aligned fingers and the sensor. The Pomodoro works only in a limited range because it uses bluetooth to interface with the robot.
We aim to overcome the limited range by using the internet instead. Viren Pereira, Vandyk Amsdem Fernandes and Junieta Sequeira [3] developed a Low Cost Object Sorting Robotic Arm using Raspberry Pi. They basically automated the process of recognising objects and sorting it using a raspberry-pi. Moreover [3] uses a basic camera module for image recognition. The paper aims to sort objects based on colors. Our idea is similar as we aim to control robots using gestures, thus the end purpose is different but the general processes and methods used are similar. From this paper, we gained a good idea about the various merits of the RPI and how we could harness the Raspberry-pi in our project. The use of the Raspberry-pi reduces the cost of the project while providing an open source and flexible Linux based platform for experimenting.
K. K. Biswas and Saurav Kumar Basu [4] conducted research on Gesture Recognition using Microsoft Kinect. They proposed a method to recognize human gestures using a Kinect depth camera. The paper explained how to obtain the depth profile of the subject and mentioned about the ability to recognize multiple human gestures using the Kinect. [4] programmed complex gestures like CLAP for Clapping, CALL for Hand gesture to call someone, WAVE for Waving hand and NO for Shaking head sideways. Thus, we understood that Kinect was one of the best gesture recognition devices available and we could use it in our project.
Patrick Benavidez and Mo Jamshidi [5] developed a Mobile Robot Navigation and Target Tracking System. They basically built an autonomous robot using an x86 based computer onboard the robot running Ubuntu Linux. Raspbian is also a flavor of Linux. Hence, the framework for the navigation of the robot was essentially the same. However, our robot is not autonomous and we will control it using hand gestures recognized by the Kinect. The problem of remote control of robotic systems has been the subject of much research in recent years. Crisman and Webb [6] developed an autonomous land vehicle for steering autonomously with high intelligence on the road under different conditions.
Varaiya [7] proposed a smart vehicle for incorporating the core driver technique into intelligent ve- hicle highway systems. Akash S A, Akshay Menon, Arpit Gupta, Md Waheeb Wakeel, Praveen MN and P. Meena [8] designed A novel strategy for controlling the movement of a Smart Wheelchair using Internet of Things. The different modes of control for the movement of the chair are made based on the specific needs of the user. Apart from the controls like joystick, chin control, voice activation, control through head movement, through calls made using a mobile, this chair can also be remotely controlled through the internet. Raspberry Pi is used as a master controller for both joystick as well as internet control. We thus realised the power of IOT (Internet of Things). IOT goes beyond machine-to-machine communications (M2M) and covers a variety of protocols, domains and applications.
