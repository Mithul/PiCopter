% Chapter 3

\chapter{REQUIREMENTS ANALYSIS} % All Chapter Headings in ALL CAPS

\section{Requirements}
\subsection{Quadcopter}
\noindent
\textbf {Hardware Requirements for the Quadcopter}
\begin{verbatim}
1) RPi 2 microcomputer
2) 9DOF IMU Sensor
3) 4 BLDC Motors
4) RPi Camera Module
5) 4 Carbon Fibre Propellers
6) Battery
7) Quadcopter Frame
8) ESC
9) Male-Female Connecting Wires
10) 4 Propeller Guards
11) Breadboard
12) Resistors
13) Portable Charger
\end{verbatim}
\textbf{Software Requirements for the Quadcopter}
\begin{verbatim}
1) Raspbian OS – Jessie
2) OpenCV 3.1.0 and OpenCV 3.1.0
3) Python
4) Android Studio
\end{verbatim}
\textbf{Functional Requirements for the Quadcopter}
\begin{verbatim}
1) Quadcopter should be connected to the WiFi network
2) LiPo batteries should be charged
3) The SD Card should be securely connected
4) The SD Card should have enough space
5) Quadcopter should move according to the processed flight control signals
\end{verbatim}
\subsection{Image Processing}
\noindent
\textbf{Hardware Requirements for the Image processing module}
\begin{verbatim}
1) RPi Camera
2) RPi microcomputer (Processing unit)
\end{verbatim}
\textbf{Software Requirements for the Image processing module}
\begin{verbatim}
1) Python 2.7.6
2) OpenCV 3.1.0
\end{verbatim}
\textbf{Functional Requirements for Image processing module}
\begin{verbatim}
1) The RPi should be connected to the WiFi network
2) The RPi Camera should be mounted securely on the quadcopter
3) The quadcopter must be stable enough to take fairly clear pictures
4) The camera signal from the Android app should be received by the quadcopter
5) The SD Card must have enough space to store the images taken
6) The RPi must execute the Python program without system failure
\end{verbatim}
\subsection{Android App Module}
\noindent
\textbf{Software Requirements for the Android App module }
\begin{verbatim}
1) Android Studio IDE
2) Terminal emulator for android
\end{verbatim}
\textbf{Hardware Requirements for the Android App module}
\begin{verbatim}
1) Android phone supporting minimum API 18
\end{verbatim}
\textbf{Functional Requirements for the Android App module}
\begin{verbatim}
1) Create the Mobile Hotspot
2) Send motion, orientation and camera signals from the app to the RPi
\end{verbatim}
\subsection{Analysis}
The quadcopter is built using the RPi. The RPi was used because of its computational power, portability and immense open-source support. The quadcopter is built with 4 motors, propellers, an IMU sensor, ESC and a RPi native camera that is interfaced with the RPi. The Android app sends JSON-encoded signals to the RPi.The RPi hosts a Python server which decodes the signals received and triggers the appropriate Python program to fly the quadcopter and take pictures. The IMU sensor detects the orientation of the quadcopter and feeds that data to the quadcopter balancing program. The program can then use the roll, pitch and yaw data to compute the required orientation of the quadcopter. The RPi Camera takes images which are processed using OpenCV image processing library on the RPi and saved on the SD Card.
\newline \newline
The next chapter describes the System Design.