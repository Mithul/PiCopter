% Chapter 7

\chapter{CONCLUSIONS} % Write in your own chapter title
\section{Contributions}
TODO
This project establishes a method to integrate the latest advances in aerial robotics and image processing that can ultimately be used for a multitude of applications in domains including aerial photogrammetry, terrain exploration and mapping, archaeological surveys, movie-making and security. The quadcopter combines the two most-often used image-processing functionalities - Panorama and 3D Reconstruction. The extreme mobility of the drone and powerful computing power of the Raspberry Pi form an effective combination for this purpose. 
\section{Future Work}
TODO
Write stuff about how the quad can be improved

The quadcopter's RPi camera currently takes pictures all at once and produces the final Panorama and 3D point cloud. This could be refined to perform the reconstruction and panorama stitching dynamically as pictures are clicked. Various options for panorama such as Polar, Cylindrical etc can be added. With regard to 3D reconstruction, the point cloud generated is a sparse 3D reconstruction of the object. This could be extended to create dense point clouds which would be more realistic.
With the integration of these new features, the drone could be used for dynamic terrain mapping, disaster relief, surveillance and tracking.