Quadcopters are one of the most popular technologies of the 21st century. They have multifarious uses ranging from aerial photogrammetry(photography and 3d positioning of objects), military surveillance, taking panorama shots, movie-making, infrastructure inspection and delivery of products. It is also an extensive area of research, an example of which would be swarm robotics. Flight stability of drones is an ongoing area of research with algorithms such as PID and Fuzzy Logic under continuous development and refinement.  
\newline
\newline
Our project delves into aerial photogrammetry i.e. taking pictures of objects/topography and reconstructing 3D representations of it. Since it is extremely difficult to take pictures or videos of places inaccessible to humans, such as caverns, waterfalls, ancient ruins, caves, etc, we can use drones to capture their images.
\newline
Our drone uses the Raspberry Pi 2 microcomputer to perform all on-board operations including flight control and image-processing. It carries an IMU sensor that detects the orientation of the quadcopter and feeds it to the PID controller. The PID controller uses this data as input to automatically stabilize the quadcopter. 
\newline  
The drone is controlled via WiFi using an Android App with a virtual joystick interface. It is also equipped with a camera module which allows it to take pictures of objects and terrain and perform panorama stitching and 3D reconstruction.